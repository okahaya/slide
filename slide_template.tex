\title[Swaq]{\color{black} \LARGE SWAQ}
\subtitle[ちょっと強いQAS]{特定の問題にちょっと強くなった量子アニーリングシミュレーター}
\author[岡田颯斗]{岡田颯斗}
% \institute[大阪府立四條畷高等学校]{大阪府立四條畷高等学校}
\date{}
\begin{frame}{}
\titlepage
\end{frame}
\large

\begin{frame}
  \frametitle{目次}
  \tableofcontents
\end{frame}

%add here

\section{自己紹介}
\begin{frame}
  \frametitle{自己紹介}
  \begin{itemize}
      \item 名前:岡田颯斗(高校3年生)
      \item 趣味・興味:
      \begin{itemize}
          \item 競技数学
          \item 量子コンピュータ
      \end{itemize}
  \end{itemize}
\end{frame}

\section{前提}
\begin{frame}
  \frametitle{前提}

  {\Large  量子アニーリングは組合せ最適化問題を解く手段の一つ}
  \vspace{5mm}

  \begin{columns}
    \begin{column}{0.45\textwidth}
      \textbf{彩色問題}
      \begin{itemize}
          \item 隣り合う場所は異なる色で塗分ける
      \end{itemize}
    \end{column}

    \begin{column}{0.45\textwidth}
      \textbf{巡回セールスマン問題}
      \begin{itemize}
          \item 複数の街を最短経路ですべて訪れる
      \end{itemize}
    \end{column}
  \end{columns}
  \vspace{5mm}
\end{frame}

\begin{frame}
  \frametitle{彩色問題の例}
  %動画のリンクへ飛ぶorローカルで流す
  彩色問題:リンク\\
  % 巡回セールスマン:リンク %デモが動き次第コメントアウトを解除
\end{frame}


\begin{frame}
  \frametitle{続:前提}

  {\Large  量子アニーリングは組合せ最適化問題を解く手段の一つ}
  \vspace{5mm}

  \begin{columns}
    \begin{column}{0.45\textwidth}
      \textbf{彩色問題}
      \begin{itemize}
          \item 隣り合う場所は異なる色で塗分ける
      \end{itemize}
    \end{column}

    \begin{column}{0.45\textwidth}
      \textbf{巡回セールスマン問題}
      \begin{itemize}
          \item 複数の街を最短経路ですべて訪れる
      \end{itemize}
    \end{column}
  \end{columns}
  \vspace{10mm}
  \pause{しかし、量子アニーリングは最適化問題を効率よく解けるかというと...}
\end{frame}

\section{背景}
\begin{frame}
  \frametitle{背景}
  {\Large 量子アニーリングはm個の中からn個選ぶのが苦手}\\
  \vspace{5mm}
  なぜなら...\\
  \begin{itemize}
    \item 制約はペナルティ項として目的関数につけられる
    \item すると問題が非本質な方向へ最適化される
  \end{itemize}

  \vspace{5mm}

  \only<1>{\[
    \begin{aligned}
        &\text{minimize} \quad H_{object} \\
        &\text{subject to} \quad H_{constraint} = c
    \end{aligned}
  \]}

  \only<2>{\[
    \begin{aligned}
        &\text{minimize} \quad H_{object} + \underbrace{(H_{constraint} - c)^2}_{H_{penalty}} \\
    \end{aligned}
  \]}

\end{frame}

\subsection{問題例}
\begin{frame}
  \frametitle{例}
    \begin{columns}[T]
      \begin{column}{0.47\textwidth}
        {\large 彩色問題}
        \footnotesize{
        \[
          \begin{aligned}
              &\text{minimize} \quad \sum_{i,j\in Adj}\sum_{k\in color}q_{i,k}q_{j,k} \\
              &\text{subject to} \quad \sum_{i\in vertics}q_{i,k} = 1 
          \end{aligned}
        \]\\
        \begin{center}
          ${\large \downarrow}$
        \end{center}
        \begin{tcolorbox}[top=0mm, left=0mm, right=0mm, bottom=0mm]
        \[
          \begin{aligned}
              \text{minimize} \quad \sum_{i,j\in Adj}\sum_{k\in color}q_{i,k}q_{j,k}\\
               + \sum_{k\in color}(\sum_{i\in Adj}q_{i,k}-1)^2\\
          \end{aligned}
        \]  
        \end{tcolorbox}
        }
      \end{column}  

      \begin{column}{0.55\textwidth}
        {\large 巡回セールスマン問題}
        \footnotesize{
        \[
          \begin{aligned}
              &\text{minimize} \quad  & & \sum_{i,j\in C}\sum_{k = 0}^n w_{i,j}q_{i,k}q_{j,k+1} \\
              &\text{subject to} \quad& & \sum_{i\in C}q_{i,k} = 1 \\
              &                       & & \sum_{k = 0}^n q_{i,k} = 1
          \end{aligned}
        \]\\
        \begin{center}
          ${\large \downarrow}$
        \end{center}
        \begin{tcolorbox}[top=0mm, left=1mm, right=0mm, bottom=0mm]
        \[
          \begin{aligned}
              &\text{minimize} \quad \sum_{i,j\in C}\sum_{k = 0}^n w_{i,j}q_{i,k}q_{j,k+1}\\
               &+ \sum_{k=0}^n(\sum_{i\in C}q_{i,k}-1)^2+\sum_{i\in C}(\sum_{k=0}^nq_{i,k}-1)^2\\
          \end{aligned}
        \]  
        \end{tcolorbox}
        }
      \end{column}  
    \end{columns}
\end{frame}

\section{デモ}
\begin{frame}
  \frametitle{デモ}
  デモをやるよ\\
  目的とみるべきポイントを紹介\\
  エネルギーが下がる様子\\
\end{frame}

\section{手法}
\begin{frame}
  \frametitle{手法}
  {\Large 制約を常に満たすように解を遷移させる}\\
  bitをswapさせます\\
  というのも...
\end{frame}

\subsection{単純な形}
\begin{frame}
  \frametitle{手法}
  単純な例について考えよう\\
  ・一次制約、二次制約、完全二次制約(造語)について触れる\\
  ・bit flipとの違い(図表で示す)
\end{frame}

\subsection{一般化}
\begin{frame}
  \frametitle{手法}
  一般化するにはごにょごにょ
\end{frame}

\section{比較}
\begin{frame}
  \frametitle{比較}
  hamiltonianにpenaltyを含むものと含まないものでのエネルギーの落ち方\\
  
\end{frame}


\section{参考文献}
\begin{frame}[t]{参考文献}
  \footnotesize

  \bibliographystyle{junsrt}
  \bibliography{bibliography.bib}
\end{frame}



