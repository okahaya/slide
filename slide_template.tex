\title[Swaq]{\color{black} \LARGE SWAQ}
\subtitle[ちょっと強いQAS]{特定の問題にちょっと強くなった量子アニーリングシミュレーター}
\author[岡田颯斗]{岡田颯斗}
% \institute[大阪府立四條畷高等学校]{大阪府立四條畷高等学校}
\date{}
\begin{frame}{}
\titlepage
\end{frame}
\large

\begin{frame}
  \frametitle{目次}
  \tableofcontents
\end{frame}

%add here

\section{自己紹介}
\begin{frame}
  \frametitle{自己紹介}
  \begin{itemize}
      \item 名前:岡田颯斗(高校3年生)
      \item 趣味・興味:
      \begin{itemize}
          \item 競技数学
          \item 量子コンピュータ
      \end{itemize}
  \end{itemize}
\end{frame}

\section{前提}
\begin{frame}
  \frametitle{前提}

  {\Large  量子アニーリングは組合せ最適化問題を解く手段の一つ}
  \vspace{5mm}

  \begin{columns}
    \begin{column}{0.45\textwidth}
      \textbf{彩色問題}
      \begin{itemize}
          \item 隣り合う場所は異なる色で塗分ける
      \end{itemize}
    \end{column}

    \begin{column}{0.45\textwidth}
      \textbf{巡回セールスマン問題}
      \begin{itemize}
          \item 複数の街を最短経路ですべて訪れる
      \end{itemize}
    \end{column}
  \end{columns}
  \vspace{5mm}
\end{frame}

\begin{frame}
  \frametitle{デモ}
  %動画のリンクへ飛ぶorローカルで流す
  彩色問題:リンク\\
  % 巡回セールスマン:リンク %デモが動き次第コメントアウトを解除
\end{frame}


\begin{frame}
  \frametitle{続:前提}

  {\Large  量子アニーリングは組合せ最適化問題を解く手段の一つ}
  \vspace{5mm}

  \begin{columns}
    \begin{column}{0.45\textwidth}
      \textbf{彩色問題}
      \begin{itemize}
          \item 隣り合う場所は異なる色で塗分ける
      \end{itemize}
    \end{column}

    \begin{column}{0.45\textwidth}
      \textbf{巡回セールスマン問題}
      \begin{itemize}
          \item 複数の街を最短経路ですべて訪れる
      \end{itemize}
    \end{column}
  \end{columns}
  \vspace{10mm}
  \pause{しかし、量子アニーリングは最適化問題を効率よく解けるかというと...}
\end{frame}

\section{背景}
\begin{frame}
  \frametitle{背景}
  {\Large 量子アニーリングはm個の中からn個選ぶのが苦手}\\
  \vspace{5mm}
  なぜなら...\\
  \begin{itemize}
    \item 制約はペナルティ項として目的関数につけられる
    \item すると問題が非本質な方向へ最適化される
  \end{itemize}

  \vspace{5mm}
  
  \only<1-2>{
  \begin{equation*}
    {H_{problem} = H_{object}} \only<2>{\color{important_font}+ H_{penalty}}
  \end{equation*}
  }

  \centering \only<1>{$H_{object}$を最適化} \only<2>{$H_{problem}+H_{penalty}$を最適化}

  \only<3>{\[
    \begin{aligned}
        &\text{minimize} \quad H_{object} \\
        &\text{subject to} \quad H_{constraint} = c
    \end{aligned}
  \]}

  \only<4>{\[
    \begin{aligned}
        &\text{minimize} \quad H_{object} + \underbrace{(H_{constraint} - c)^2}_{H_{penalty}} \\
    \end{aligned}
  \]}

\end{frame}

\begin{frame}
  \frametitle{背景}
  hogehoge
  %グラフ彩色問題での実験結果を載せる
  %onehotありなし
\end{frame}

\section{手法}
\begin{frame}
  \frametitle{手法}
  {\Large 制約を常に満たすように解を遷移させる-Swap Based}

\end{frame}

\section{参考文献}
\begin{frame}[t]{参考文献}
  \footnotesize

  \bibliographystyle{junsrt}
  \bibliography{bibliography.bib}
\end{frame}



